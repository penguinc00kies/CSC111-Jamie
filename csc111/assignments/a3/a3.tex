\documentclass[11pt]{article}
\usepackage{amsmath}
\usepackage[utf8]{inputenc}
\usepackage[margin=0.75in]{geometry}

\title{CSC111 Assignment 3: Graphs, Recommender Systems, and Clustering}
\author{Jamie Yi}
\date{\today}

\begin{document}
\maketitle

\section*{Part 1: The book review graph and simple recommendations}

\begin{enumerate}

\item[1.]
Complete this part in the provided \texttt{a3\_part1.py} starter file.
Do \textbf{not} include your solution in this file.

\item[2.]
Let $m$ be the number of lines in the \texttt{reviews\_file} file. \\
Let $n$ be the number of lines in the \texttt{book\_names\_file} file. \\

The first two assignment statements are both constant time, 1 step. \\
\texttt{open} and \texttt{csv.reader} are both constant time, 1 step. \\
The first for loop iterates exactly $n$ times and only contains an assignment statement which is constant time, so the whole loop will take $n(1)$ or just $n$ steps. \\
\texttt{open} and \texttt{csv.reader} are both constant time, 1 step. \\
The comprehension will iterate exactly $m$ times, and creating a set and appending it to a list are both constant time, so the comprehension will take $m(1)$ or just $m$ steps. \\

The second for loop will iterate exactly $m$ times, and all the statements inside it are constant time, so the whole loop will take $m(1)$ or just $m$ steps.\\
\texttt{Graph.add\_vertex} only has one assignment statement and its if statement condition are both constant time, 1 step.\\
\texttt{Graph.add\_edge} has assignment statements, \texttt{set.add}(both of which are constant time) and its if statement condition is both constant time, 1 step.\\

The final return statement is constant time, 1 step.\\

The function will run $1 + 1 + n + 1 + m + m + 1$ steps per call. Which is equal to $n + 2m + 4$ Therefore we have a running time of $\Theta(m + n)$ \\


\item[3.]
Complete this part in the provided \texttt{a3\_part1.py} starter file.
Do \textbf{not} include your solution in this file.

\item[4.]
Complete this part in the provided \texttt{a3\_part1.py} starter file.
Do \textbf{not} include your solution in this file.

\end{enumerate}

\section*{Part 2: Weighted graphs, recommendations, review prediction}

Complete this part in the provided \texttt{a3\_part2\_recommendations.py} and \texttt{a3\_part2\_predictions.py} starter files.
Do \textbf{not} include your solution in this file.

\newpage

\section*{Part 3: Finding book clusters}

\begin{enumerate}

\item[1.]
Complete this part in the provided \texttt{a3\_part3.py} starter file.
Do \textbf{not} include your solution in this file.

\item[2.]
Complete this part in the provided \texttt{a3\_part3.py} starter file.
Do \textbf{not} include your solution in this file.

\item[3.]

\begin{enumerate}
\item[(a)]
Let $m_1$ be the size of \texttt{cluster1} \\
Let $m_2$ be the size of \texttt{cluster2} \\

The first two assignment statements are both constant time, 1 step. \\
The outer for loop will iterate exactly $m_1$ times. \\
The inner for loop will iterate exactly $m_2$ times, the assignment statement inside the inner for loop takes 1 step constant time, since the \texttt{Graph.get\_weight} method is constant time (it contains assignment statements and dictionary indexing, both of which are always constant time). \\
Therefore the whole loop will take $m_1(m_2(1))$ steps, or just $m_1m_2$ steps. \\
The final return statement is constant time, 1 step. \\

The function will run $1 + m_1m_2 + 1$ steps per call. Which is equal to $m_1m_2 + 2$ Therefore we have a running time of $\Theta(m_1m_2)$ \\

\item[(b)]
Let $n$ be the number of vertices in \texttt{graph}. \\
For any fixed iteration of the outer loop, \texttt{clusters} will be at most length $n$, containing the clusters $c_1, c_2, c_3, ..., c_n$. \\
Let the notation $size(c_0)$ represent the size of $c_0$. \\
Fix an arbitrary $c1$, such that we can treat $size(c1)$ as a constant. \\

The for loop will iterate at most $n$ times. \\
For every iteration except when \texttt{c1 is c2}, the loop will run $size(c1) \cdot size(c2) + 1$ steps (we know the running time of \texttt{cross\_cluster\_weight} from part a). Since sum of all cluster sizes is equal to $n$ and \texttt{c1 is not c2}, then at most $size(c2)$ will equal $(n - s(c1))$. \\
When \texttt{c1 is c2}, the loop will run 1 step to check the if statement. \\
Assuming the loop iterates the maximum amount of $n$ times, it will take at most $1 + (n - 1)(n - s(c1))$ steps. \\

The loop will run at most $1 + (n - 1)(n - size(c1))$ steps per call at most, which can be rewritten as $n^2 - size(c1)n - n + size(c1)$ steps. Therefore we have a running time of $\mathcal{O}(n^2)$ (since we treat $size(c1)$ as a constant) \\


\item[(c)]
Let $n$ be the number of vertices in \texttt{graph}. \\
Let $k$ be the is the value of \texttt{num\_clusters}. \\

The first comprehension is creating a set and appending it to a list which is constant time (the method \texttt{Graph.get\_all\_vertices} is also constant time if it has no arguments passed to it). It will iterate exactly $n$ times so the comprehension will take $n(1)$ or just $n$ steps. \\

Assuming no clusters are ever merged in the for loop, the outer for loop will iterate at most $(n - k)$ times. \\
The first 4 statements (print, \texttt{random.choice}, and assignment) are all 1 step constant time. \\
As shown in part b, the inner for loop will run for at most $n^2 - size(c1)n - n + size(c1)$ steps. \\
The value for any given \texttt{c1} is at most $n - size(c1)$ (since there should be at least 2 clusters in \texttt{clusters} while the loop is running), so the \texttt{set.update} statement will take at most $n - size(c1)$ steps. \\
\texttt{set.remove} is constant time, so it's 1 step constant time.\\
Therefore, the loop will run $4 + n^2 - size(c1)n - n + size(c1) + (n - size(c1)) + 1$ at most per iteration. \\

The final return statement is constant time, 1 step.\\

The function will run $n + (n - k)(4 + n^2 - size(c1)n - n + size(c1) + (n - size(c1)) + 1)$ steps per call at most. Therefore we have a running time of $\mathcal{O}(n^2(n - k))$ \\

\item[(d)]
\emph{Not to be handed in.}
\end{enumerate}

\end{enumerate}
\end{document}
